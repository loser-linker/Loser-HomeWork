日期:2023/7/21 出题人:mq白\\

给出以下代码,在不修改已给出代码的前提下使它满足\textbf{运行结果}。

\begin{minted}[mathescape,	
    linenos,
    numbersep=5pt,
    gobble=2,
    frame=lines,
    framesep=2mm]{c++}
    int main(){
        std::vector v{1, 2, 3};
        std::function f {[](const int& i) {std::cout << i << ' '; } };
        auto f2 = [](int& i) {i *= i; };
        v | f2 | f;
    }
\end{minted}

\begin{tcolorbox}[title = {要求运行结果},
        fonttitle = \bfseries, fontupper = \sffamily, fontlower = \itshape]
    1 4 9
\end{tcolorbox}

\begin{itemize}
    \item \textbf{难度}: \hardscore{2} \\ 
    \textbf{提示}:\mintinline{c++}{T& operator|(T& v, const T& f) }。
\end{itemize}

\subsection{答案}

\begin{minted}[mathescape,	
    linenos,
    numbersep=5pt,
    gobble=2,
    frame=lines,
    framesep=2mm]{c++}
    template<typename U, typename F>
        requires std::regular_invocable<F, U&>//可加可不加,不会就不加
    std::vector<U>& operator|(std::vector<U>& v1, F f) {
        for (auto& i : v1) {
            f(i);
        }
        return v1;
    }
\end{minted}

\textbf{不使用模板:}

\begin{minted}[mathescape,	
    linenos,
    numbersep=5pt,
    gobble=2,
    frame=lines,
    framesep=2mm]{c++}
    std::vector<int>& operator|(std::vector<int>& v1, const std::function<void(int&)>& f) {
        for (auto& i : v1) {
            f(i);
        }
        return v1;
    }
\end{minted}

\textbf{不使用范围 for,使用 C++20 简写函数模板:}

\begin{minted}[mathescape,	
    linenos,
    numbersep=5pt,
    gobble=2,
    frame=lines,
    framesep=2mm]{c++}
    std::vector<int>& operator|(auto& v1, const auto& f) {
        std::ranges::for_each(v1, f);
        return v1;
    }
\end{minted}

\textbf{各种其他答案的范式无非就是这些改来改去了,没必要再写。}

\subsection{解析}

很明显我们需要重载管道运算符 $|$,根据我们的调用形式 v $|$ f2 $|$ f,
这种\textbf{链式}的调用,以及根据给出运行结果,我们可以知道,重载函数应当返回 v 的引用,并且 v 会被修改。

v $|$ f2 调用 \mintinline{c++}{operator |},operator $|$ 中使用 f2 遍历了 v 中的每一个元素,然后返回 v 的引用,再 $|$ f。

形式和原理很简单,那么接下来就是实现;最简单的方式无非就是写一个模板

\begin{minted}[mathescape,	
    linenos,
    numbersep=5pt,
    gobble=2,
    frame=lines,
    framesep=2mm]{c++}
    template<typename T, typename F>
    T& operator|(T& v1, F f) {
        for (auto& i : v1) {
            f(i);
        }
        return v1;
    }
\end{minted}

当然了,这个模板还不够好,我们知道了第一个参数会是 vector,模板完全可以再准确一点:

\begin{minted}[mathescape,	
    linenos,
    numbersep=5pt,
    gobble=2,
    frame=lines,
    framesep=2mm]{c++}
    template<typename U, typename F>
    std::vector<U>& operator|(std::vector<U>& v1, F f)
\end{minted}

考虑到 std::function 的复制开销也不小,第二个模板形参也可以加 const\&。

\href{https://zh.cppreference.com/w/cpp/language/range-for}{范围 for},以及 \href{https://zh.cppreference.com/w/cpp/language/requires}{requires} 不再介绍。

\clearpage